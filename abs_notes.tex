\documentclass{article}
\usepackage[utf8]{inputenc}

\title{ABS Notes}
\author{james.friel4 }
\date{May 2016}

\begin{document}

\maketitle

\section{Pervasive Trends In Computing}
\begin{itemize}
    \item Ubiquity - Cost of processing power decreasing
    \item Interconnection - System interaction
    \item Complicity - Elaboration of tasks carried out by computers
    \item Delegation - Giving control to a computer
    \item Human-orientation - Increase use of metaphors that better reflect human intuition
\end{itemize}

\section{What Is an Agent?}
An agent id anything that can perceive its environment(through its sensors) and act upon that environment(through its effectors)

\section{Environment Classification}
\begin{itemize}
    \item Accessible vs Inaccessible
    \item Deterministic vs Non-deterministic
    \item Static vs Dynamic
    \item Episodic vs Non-episodic
    \item Discrete vs Continuous
    \item Open Environment
\end{itemize}

\section{Abstract Agents}
Run = sequence of interleaved environment states and actions
$ r: e_0 \rightarrow e_1 \rightarrow e_2 ... e_{u-1}\rightarrow e_u$
R = {r,r',...} the set of all possible runs
\subsection{State Transformer Function}
$\tau: R^{Ac} \rightarrow \varphi(E)
\tau $maps each run ending with an agent action to the set of possible resulting states
if $\tau(r) = \theta$ then the system will terminate

Agents are behavioural equivalent if R(Ag1,Env) = R(Ag2,Env).


\section{Reactive Agents}
Bases decisions only on current state.
Every reactive agent can be mapped to an agent defined on runs( the reverse is not usually true).

Ag: E $\rightarrow$ Ac
An environment state maps to an agent action.

\section{Perception and Action}
see:  E: $\rightarrow$ Per
action:  Per* $\rightarrow Ac$
Where Per is a non-empty set of perceptions and actions defines decisions based on perceptions.
Two perceptions are indistinguishable if see(e1) = see(e2)

\section{State Based Agents}
Think FSM with states:
see: E \rightarrow Per
action: I \rightarrow Ac
next: I x Per \rightarrow I

\section{Utilities}
Utilites describe "quality" of a state through some numerical value
u: E $\rightarrow$ R
This makes the long-term view difficult to account for, can use runs instead

Using utilities we can make optimal agents,
"An optimal agent is one that maximises expected utility"
define P(r|Ag,Env) is the probability that a run occurs given an agent, Ag, and an environment, Env.
So, Ag_{opt} = argmax $\Sigma$ P(r| Ag,Env)u(r)


\section{MetateM}
Concurrent MetateM is based on direct execution of logical formulae.
Concurrently executing agents communicate via message passing.
Look up the tutorial with Snow White to get an Idea about this.

\section{Practical Reasoning Agents}
Reasoning towards actions (deciding what to do)
\begin{itemize}
    \item Deliberation: deciding what to do
    \item Means-end reasoning: deciding how to do it
\end{itemize}
Combining these is the foundation of deliberative agency.
Deliberation generates intentions, Means-end generates plans.
\subsection{BDI Architecture}
\begin{itemize}
    \item Belief Revision Function
    \item Generate Options
    \item Filtering Function
    \item Planning Function
    \item Action Generator
\end{itemize}

\section{Subsumption Architecture}
See Mars Rover Example

\section{InteRRaP}
InteRRaP: Integrated of rational planning and reactive behaviour
Has a vertical layering architecture.
Has 3 layers:
\begin{itemize}
    \item Behaviour-base Layer
    \item Local Planning Layer
    \item Social Planning Layer
\end{itemize}

\section{Agent Interaction}
\begin{itemize}
    \item Non-/quasi-communicative interaction - shared environment
    \item Communication - info exchange, collaboration, negotiation
\end{itemize}
Most multiagent approaches to communication based on speech act theory,
consists of Locution, Illocution, Perlocution.

Common communication laguagews are KQML and FIPA
Look up the exact working of these languages - there's loads

\section{Agent Coordination}
\subsection{Relationships}
Positive relationships are between two agents where at least one agent benefits if the plans are combined.
Requests: Explicitly asking for help with own activities

\subsection{Partial Global Planning}
PGP: exchange information to reach common conclusion
Three iterated stages:
\begin{itemize}
    \item Agents deliberate locally and generate short-term plans for goal achievement
    \item They exchange information to determine where plans and goals interact
    \item Agents alter local plans to better coordinate their activities   
\end{itemize}

\section{Agent Preferences}
Preference ordering OxO for agent i is a total, antisymmetric, transitive relation.
Preferences are often expressed through a utility function.
\subsection{Game Theory}
Some games have a dominant strategy equilibrium - all agents have a dominating strategy

\subsection{Nash Equilibrium}
A join strategy S is said to me a Nash equilibrium if no agent has a incentive to deviate from this strategy combination.

\section{Social Choice}
\subsection{Preference Aggregation}
How do we combine a collection of potentially different preference orders in order to derive a group decision?
Voters submit orderings of preferences, the outcome that appears first in most orderings wins.

Condorcet’s Paradox:
There are scenarios in which no matter which outcome we choose the majority of voters will be unhappy with the outcome chosen

\subsection{Borda Count}
only top-ranked candidate taken into account, rest of orderings disregarded.
Borda count looks at entire preference ordering, counts the strength of opinion in favour of a candidate

\section{Coalitions}
The three stages of Cooperative action:
\begin{itemize}
    \item Coalition structure generation
    \item Solving the optimisation problem of each coalition
    \item Dividing the value of the solution of each coalition
\end{itemize}
The core of a coalitional game is the set of outcomes that no sub-coalition can object to.

\subsection{Shapley Value}
\begin{itemize}
\item Symmetry: if two agents contribute the same they should receive the same pay-off (they are interchangeable)
\item Dummy player: agents that do not add value to any coalition should get what they earn on their own
\item Additivity: if two games are combined, the value a player gets should be the sum of the values it gets in individual games
\end{itemize}
$sh_i = \frac{1}{|Ag|!}\Sigma_o  \varepsilon \pi(Ag) \mu i^{C_i(o)}$

\section{Auctions}
\begin{itemize}
\item English
\item Dutch
\item Vickrey
\end{itemize}

\section{Bargaining}
I'll get back to this

\section{Argumentation}
...
\end{document}

